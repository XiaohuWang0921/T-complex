\documentclass{article}
\usepackage{amsthm,amssymb,amsmath,tikz-cd,verbatim,hyperref,stmaryrd}
\newtheorem{theorem}{Theorem}[subsection]
\newtheorem{lemma}{Lemma}[subsection]
\newtheorem{corollary}{Corollary}[subsection]
\theoremstyle{remark}
\newtheorem{remark}{Remark}[subsection]
\theoremstyle{definition}
\newtheorem{definition}{Definition}[subsection]
\newcommand{\N}{\mathbb N}
\newcommand{\Fin}{\mathrm{Fin}}
\newcommand{\Set}{\mathrm{Set}}
\newcommand{\PSF}{\mathrm{PSF}}
\newcommand{\op}{\mathrm{op}}
\newcommand{\p}{\mathfrak P}
\newcommand{\ASC}{\mathsf{ASC}}
\newcommand{\im}{\mathrm{im}}
\newcommand{\id}{\mathrm{id}}
\newcommand{\Y}{\mathcal Y}
\newcommand{\X}{\mathcal X}
\newcommand{\K}{\mathcal K}
\newcommand{\g}{\mathfrak g}
\newcommand{\F}{\mathfrak F}
\newcommand{\Tcplx}{\mathrm{Tcplx}}
\newcommand{\Q}{\mathcal Q}
\newcommand{\M}{\mathcal M}
\newenvironment{example}{\begin{proof}[Example]}{\end{proof}}
\begin{document}
	\section{T-complexes}
	\subsection{Canonical Finite Sets}
	\begin{definition}[Canonical finite sets]
		For each $n\in\N$, let $[n]:=\{m\in\N\mid m<n\}$ be the \textit{$n$-th canonical finite set}.
	\end{definition}
	\begin{definition}[Category of nonempty canonical finite sets]\label{fin+}
		Let $\Fin_+$ denote the full subcategory of $\Set$ generated by all nonempty canonical finite sets, i.e. all $[n]$ with $n>0$.
	\end{definition}
	\begin{lemma}\label{prodsum}
		For $m,n\in\N$, the function
		\[p_{m,n}:[m]\times[n]\to[m\cdot n],\quad(a,b)\mapsto an+b\]
		is bijective, as well as
		\[q_{m,n}:[m]\uplus[n]\to[m+n],\quad\begin{cases}
			(0,i)&\mapsto i\\
			(1,j)&\mapsto m+j
		\end{cases}\]
		The inverse of $p_{m,n}$ is simply the Euclidean division by $n$, while the inverse of $q_{m,n}$ is the following:
		\[r_{m,n}:[m+n]\to[m]\uplus[n],\quad k\mapsto\begin{cases}
			(0,k)&k<m\\
			(1,k-m)&k\geq m
		\end{cases}\]
	\end{lemma}
	\begin{proof}
		Exercise for reader.
	\end{proof}
	\begin{corollary}
		$\Fin_+$ as defined in Definition \ref{fin+} has finite products (in fact even finitely complete) as well as finite nonempty sums. The nullary product is $[1]$ and for each $[m]$ and $[n]$, the binary product is $[m\cdot n]$, while the binary sum is $[m+n]$, with the encoding demonstrated by Lemma \ref{prodsum}.
	\end{corollary}
	\begin{proof}
		Exercise for reader.
	\end{proof}
	For convenience, we shall write $p_{m,n}(a,b)$ simply as $(a;b)$ whenever $m$ and $n$ are clear from context. It resembles the common notation for ordered pairs, thus reflecting its similar purpose to the set-theoretic cartesian product.
	\begin{definition}[Global and local elements]
		Because $\Set$ is cartesian closed with $[1]$ being the terminal object, there is the following natural isomorphism between two endofunctors $\Set\to\Set$:
		\[\g_A:A\xrightarrow\sim\Set([1],A)\]
		For each $x\in A$, it can be explicitly defined as
		\[\g_A(x):[1]\to A,\quad 0\mapsto x\]
		Which we shall call the \textit{global element} for $x$. Conversely, for each $f:[1]\to A$, we shall call $f(0)\in A$ its corresponding \textit{local element}.
		
		By abuse of notation, we shall sometimes consider $\g$ to be a natural isomorphism between functors of type $\Fin_+\to\Set$ instead, and write
		\[\g_n:[n]\xrightarrow\sim\Fin_+([1],[n])\]
		for every $n\in\N_+$.
	\end{definition}
	\subsection{Abstract Simplicial Complexes}
	\begin{definition}[Abstract simplicial complex]
		For each set $X$, a $C\subset\p(X)$ is called an \textit{abstract simplicial complex (ASC for short) over $X$} if it satisfies three conditions
		\begin{enumerate}
			\item It is downward closed, i.e. whenever we have $S\subset T\in C$, we also have $S\in C$.
			\item It contains all singletons subsets of $X$, i.e. $\forall x\in X:\{x\}\in C$.
			\item It contains the empty set $\emptyset$ (which ensures that $C$ is not empty).
		\end{enumerate}
		Let $\ASC_n$ denote the set of all ASC over $[n]$.
	\end{definition}
	\begin{example}
		\begin{enumerate}
			\item For every set $X$, $C_0:=\p(X)$ is the greatest ASC over $X$ with respect to inclusion, and $C_1:=\{\emptyset\}\cup\left\{\{x\}\mid x\in X\right\}$ the smallest.
			\item $C_2:=\{S\subset X\mid S\neq X\}$ is an ASC over $X$ if $|X|>1$.
			\item $C_3:=\{S\subset \N\mid S\text{ is finite}\}$ is an ASC over $\N$.
		\end{enumerate}
	\end{example}
	\begin{definition}[Center of an ASC]
		Let $C$ be an ASC over some set $X$. Let $Z(C):=\{x\in X\mid\forall S\in C: \{x\}\cup S\in C\}$ denote the \textit{center} of $C$. If $Z(C)\neq\emptyset$, we call $C$ \textit{centralized}.
	\end{definition}
	\begin{example}
		Here we take a look at the center of each example of Definition 1.2.1
		\begin{enumerate}
			\item $Z(C_0)=Z(\p(X))=X$, $Z(C_1)=\emptyset$ if $|X|\neq1$, otherwise $Z(C_1)=\{x\}$ if $X=\{x\}$.
			\item $Z(C_2)=\emptyset$
			\item $Z(C_3)=\N$
		\end{enumerate}
	\end{example}
	\begin{theorem}
		Let $C\in\ASC_n$. Then $Z(C)\in C$.
	\end{theorem}
	\begin{proof}
		Per definition we have $\emptyset\in C$. $Z(C)\in C$ then follows by induction over $|Z(C)|$.
	\end{proof}
	\begin{remark}
		Note that Lemma 1.2.1 fails when we consider ASC over an infinite set. In Example 3 of Definition 1.2.2 we see that $Z(C_3)=\N\notin C_3$.
	\end{remark}
	\begin{definition}[Generated ASC]
		Let $\mathcal T\subset\p(X)$ be a system of subsets of $X$. We call the ASC $\langle\mathcal T\rangle:=\{S\subset X\mid\exists T\in\mathcal T:S\subset T\}\cup\{\emptyset\}\cup\left\{\{x\}\mid x\in X\right\}$ to be \textit{generated by $\mathcal T$}.
	\end{definition}
	\begin{example}
		\begin{enumerate}
			\item $\left\langle\{X\}\right\rangle=\p(X)$ and $\langle\emptyset\rangle=\{\emptyset\}\cup\left\{\{x\}\mid x\in X\right\}$
			\item $\left\langle X\setminus\{x\}\mid x\in X\right\rangle=\{S\subset X\mid S\neq X\}$ if $|X|>1$.
		\end{enumerate}
	\end{example}
	The following lemma justifies the name:
	\begin{lemma}
		$\langle\mathcal T\rangle$ is the smallest ASC that contains $\mathcal T$, i.e. for every ASC $C$ with $\mathcal{T}\subset C$, we have $\langle\mathcal T\rangle\subset C$.
	\end{lemma}
	\begin{proof}
		Suppose we have an ASC $C$ with $\mathcal T\subset C$, and let $S\in\langle\mathcal{T}\rangle$. Then $S=\emptyset$ or $S=\{x\}$ for some $x\in X$, in which case clearly $S\in C$; or there exists some $T\in\mathcal{T}$ s.t. $S\subset T$. Hence we have $S\subset T\in\mathcal{T}\subset C$, therefore $S\in C$.
	\end{proof}
	\begin{definition}
		Let $f:X\to Y$ be a function and $C$ an ASC over $Y$. We call $C^f:=\{S\subset X\mid f[S]\in C\}$ the ASC of preimages.
	\end{definition}
	\begin{example}
		Again, we take a look at the examples introduced after Definition 1.2.1
		\begin{enumerate}
			\item $\p(X)^f=\p(Y)$ and $\left(\{\emptyset\}\cup\left\{\{x\}\mid x\in X\right\}\right)^f=\{\emptyset\}\cup\left\{\{y\}\mid y\in Y\right\}$ for every $f:Y\to X$.
			\item If $f:Y\to X$ is not surjective, then $C_2^f=\p(Y)$.
			\item For every $f:Y\to\N$, $C_3^f=\{S\subset Y\mid S\text{ is finite}\}$. In particular, if $Y$ is finite, then $C_3^f=\p(Y)$; if $Y=\N$, then $C_3^f=C_3$.
		\end{enumerate}
	\end{example}
	Again, the following lemma justifies the name:
	\begin{lemma}
		For $f:X\to Y$ and $C$ an ASC over $Y$, $C^f=\left\langle f^{-1}[S]\mid S\in C\right\rangle$.
	\end{lemma}
	\begin{proof}
		According to Lemma 1.2.2, we can simply prove that $C^f$ is the smallest ASC that contains all preimages $f^{-1}[S]$ with $S\in C$.
		
		Firstly, it is clear that $C^f$ itself contains all such preimages, since $f\left[f^{-1}[S]\right]\subset S\in C$.
		
		Secondly, for every ASC $D$ over $X$ that contains all such preimages $f^{-1}[S]$ with $S\in C$, and for every $T\in C^f$, we have $f[T]\in C$, hence $T\subset f^{-1}\left[f[T]\right]\in D$, therefore $T\in D$.
	\end{proof}
	\begin{theorem}
		Let $f:X\twoheadrightarrow Y$ be a surjection and let $C$ be an ASC over $Y$. Then $Z\left(C^f\right)=f^{-1}\left[Z(C)\right]$. In particular, if $C$ is centralized, then so is $C^f$.
	\end{theorem}
	\begin{proof}
		Let $x\in Z\left(C^f\right)$, and let $T\in C$. Since $f$ is surjective, we have $f\left[f^{-1}[T]\right]=T$. Furthermore, $\{x\}\cup f^{-1}[T]\in C^f$, which means
		\[f\left[\{x\}\cup f^{-1}[T]\right]=f\left[\{x\}\right]\cup f\left[f^{-1}[T]\right]=\left\{f(x)\right\}\cup T\in C\]
		Since $T$ was chosen arbitrarily, this shows that $f(x)\in Z(C)$, or $x\in f^{-1}\left[Z(C)\right]$. Since $x$ was also chosen arbitrarily, this shows that $Z\left(C^f\right)\subset f^{-1}\left[Z(C)\right]$.
		
		Conversely, whenever we have $x\in f^{-1}\left[Z(C)\right]$, i.e. $f(x)\in Z(C)$, and $S\in C^f$, we have
		\[f\left[\{x\}\cup S\right]=\left\{f(x)\right\}\cup f[S]\in C\]
		Hence $\{x\}\cup S\in C^f$. Again, because $x$ was chosen arbitrarily, we have $f^{-1}\left[Z(C)\right]\subset Z\left(C^f\right)$.
	\end{proof}
	\begin{corollary}\label{centerpi}
		Let $X,Y$ be lets, $X$ nonempty, $C$ an ASC over $Y$. Let $\pi^2:X\times Y\twoheadrightarrow Y$ denote the projection from the cartesian product. Then $Z\left(C^{\pi^2}\right)=X\times Z(C)$.
	\end{corollary}
	\begin{proof}
		$\pi^2$ is surjective if $X$ is nonempty. Theorem 1.2.2 then delivers the specified equality.
	\end{proof}
	\subsection{(Concrete) Simplicial Complexes}
	From here on, we will drop the brackets and simply write $[n]$ as $n$ whenever a natural number is used where a set is expected.
	\begin{definition}
		Let $\PSF$ be the category of presheaves over $\Fin_+$, i.e. $\Set^{\Fin_+^\op}$, and let $\Y:\Fin_+\hookrightarrow\PSF$ denote the Yoneda-embedding of $\Fin_+$. Furthermore, for every $n\in\N$ and $C\in\ASC_n$, let $\X(C)$ denote the subpresheaf of $\Y(n)$ consisting of only functions $f:m\to n$ with $\im(f)\in C$. Let the inclusion $\X(C)\subset\Y(n)$ be denoted by $\iota(C):\X(C)\hookrightarrow\Y(n)$. Clearly, $\Y(n)=\X(\p(n))$.
	\end{definition}
	\begin{definition}[Simplexes and (concrete) simplicial complexes]
		Let $n\in\N_+$ and $C\in\ASC_n$, and $P:\Fin_+^\op\to\Set$ be a presheaf over $\Fin_+$. A natural transformation $s:\Y(n)\to P$ is called an \textit{$n$-simplex} in $P$, while a $c:\X(C)\to P$ is called a \textit{(concrete) simplicial complex of shape $C$} in $P$.
	\end{definition}
	\begin{remark}
		Yoneda lemma asserts $\PSF(\Y(n),P)\cong P(n)$. Hence one could alternatively define an $n$-simplex in $P$ simply as a $s\in P(n)$.
	\end{remark}
	\begin{definition}
		For $m,n\in\N$ and $f:m\to n$, $\Y(f):\Y(m)\to\Y(n)$ is a natural transformation. Analogously, let $\X(f):\X(C^f)\to\X(C)$ also denote the natural transformation defined by composition with $f$, i.e.
		\[\X(f):\X(C^f)\to\X(C),\quad g\mapsto f\circ g\]
	\end{definition}
	\begin{lemma}
		The following square is a pullback:
		\[\begin{tikzcd}
			&\X(C^f)\arrow{r}{\X(f)}\arrow[hook]{d}[swap]{\iota(C^f)}&\X(C)\arrow[hook]{d}{\iota(C)}\\
			&\Y(m)\arrow{r}[swap]{\Y(f)}&\Y(n)
		\end{tikzcd}\]
	\end{lemma}
	\begin{proof}
		$\im(g)\in C^f$ if and only if $\im(f\circ g)\in C$, therefore $\X(C^f)\subset\Y(m)$ is essentially the preimage of $\X(C)\subset\Y(n)$ under $\Y(f)$.
	\end{proof}
	\begin{theorem}[Simplicial complexes in $\Y(n)$]\label{lift}
		Every simplicial complex in $\Y(n)$, i.e. every $c:\X(C)\to\Y(n)$ with $C\in\ASC_m$, can be lifted into a simplex $s:\Y(m)\to\Y(n)$ in a unique way, i.e. there is a unique $s:\Y(m)\to\Y(n)$ s.t. the following triangle commutes:
		\begin{equation}\label{Lift}
			\begin{tikzcd}
				\X(C)\arrow{r}{c}\arrow{d}[swap]{\iota_C}&\Y(n)\\
				\Y(m)\arrow{ur}[swap]{s}
			\end{tikzcd}
		\end{equation}
	\end{theorem}
	\begin{proof}
		Suppose such an lifting simplex exists, then per Yoneda-lemma it must arise from some $f:m\to n$ with $s=\Y(f)$. For each $i\in[m]$, $\im(\g_m(i))=\{i\}\in C$, so $\g_m(i)\in\X(C)(m)$. So since $c=s\circ\iota_C$, we must have
		\[s_1(\g_m(i))=s_1(\iota_{C1}(\g_m(i)))=c_1(\g_m(i))\]
		Under the Yoneda-correspondence, $s_l(h)=f\circ h$ for every $h:l\to m$, so the above equation becomes
		\[f\circ\g_m(i)=c_1(\g_m(i))\]
		Both sides are in fact functions of type $1\to n$, so applying both sides to $0\in[1]$, we get
		\[f(\g_m(i)(0))=c_1(\g_m(i))(0)\]
		But $\g_m(i)(0)$ is per definition simply $i$, so the left-hand side just becomes $f(i)$:
		\begin{equation}\label{functional}
			f(i)=c_1(\g_m(i))(0)
		\end{equation}
		So we've shown that if the lifting simplex $s:\Y(m)\to\Y(n)$ exists, it must arise from some $f:m\to n$ satisfying equation \ref{functional}. On the other hand, equation \ref{functional} also uniquely defines $f$, and one can easily verify that $s=\Y(f)$ is indeed lifts $c$.
	\end{proof}
	\begin{remark}
		Theorem \ref{lift} unfortunately does not generalize to simplicial complexes in $\X(C)$ for arbitrary $C\in\ASC_n$, even if $C$ is centralized. This can be seen by the fact that the identity transformation $\id_{\X(C)}:\X(C)\to\X(C)$ is a simplicial complex in $\X(C)$, however it has no lifting simplex $\Y(n)\to\X(C)$ if $C\neq\p(n)$, since any such lifting simplex must send $\id_n\in\Y(n)(n)$ to itself, which is not in $\X(C)(n)$.
	\end{remark}
	\begin{definition}
		For a simplicial complex $c:\X(C)\to\Y(n)$, let the unique lifting simplex delivered by Theorem \ref{lift} be denoted by $\F(c):\Y(m)\to\Y(n)$. Using equation \ref{functional}, it can be explicitly defined by
		\[\F(c):=\Y(\g_n^{-1}\circ c_1\circ\g_m)\]
	\end{definition}
	It is trivial to show that if $c:\X(C)\twoheadrightarrow\Y(n)$ is an epimorphism, then so is $\F(c):\Y(m)\twoheadrightarrow\Y(n)$, simply because $c=\F(c)\circ\iota_C$. This fact will probably never be used in proofs, but we will nonetheless reference it in our diagrams.
	\begin{definition}[T-complexes]\label{Tcplx}
		A \textit{T-complex} consists of an ordered pair $(P,\phi)$ where:
		\begin{enumerate}
			\item $P:\Fin_+^\op\to\Set$ is a presheaf over $\Fin_+$;
			\item $\phi$ is a family of simplexes, with $\phi^s:\Y(n)\to P$ for each centralized $C\in\ASC_n$ and each simplicial complex $s:\X(C)\to P$, that lifts $s$ i.e. $s=\phi^s\circ\iota(C)$, called the \textit{filler} for $s$. In diagram:
			\[\begin{tikzcd}
				&\X(C)\arrow{r}{s}\arrow[hook]{d}[swap]{\iota(C)}&P\\
				&\Y(n)\arrow{ur}[swap]{\phi^s}
			\end{tikzcd}\]
		\end{enumerate}
		The fillers are subject to the following two axioms, for every $C\in\ASC_n$ centralized:
		\begin{enumerate}
			\item For every $f:m\to n$ s.t. $C^f$ is also centralized, the following diagram, in particular the lower triangle, commutes:
			\begin{equation}\label{pullbackcomp}\begin{tikzcd}
				&\X(C^f)\arrow[hook]{dd}[swap]{\iota(C^f)}\arrow{rr}{\X(f)}&&\X(C)\arrow[hook]{ld}[swap]{\iota(C)}\arrow{dd}{s}\\
				&&\Y(n)\arrow{dr}{\phi^s}\\
				&\Y(m)\arrow{ur}{\Y(f)}\arrow{rr}[swap]{\phi^{s\circ\X(f)}}&&P
			\end{tikzcd}\end{equation}
			Summarized in an equation, this just says $\phi^{s\circ\X(f)}=\phi^s\circ\Y(f)$.
			\item For $c:\X(C)\twoheadrightarrow\Y(m)$ an epic centralized simplicial complex in $\Y(m)$ and $s:\Y(m)\to P$ an $m$-simplex in $P$, then $\phi^{s\circ c}$ factors through $\F(c)$. So the following diagram commutes:
			\begin{equation}\label{epifac}\begin{tikzcd}
				\X(C)\arrow[two heads]{r}{c}\arrow[hook]{dr}[swap]{\iota_C}&\Y(m)\arrow{r}{s}&P\\
				&\Y(n)\arrow[two heads]{u}{\F(c)}\arrow{ur}[swap]{\phi^{s\circ c}}
			\end{tikzcd}\end{equation}
		\end{enumerate}
		As is common with algebraic structures, we shall denote a T-complex simply $(P,\phi)$ simply by its underlying presheaf $P$, whenever the fillers can be inferred from the context. It should be kept in mind though that the fillers are really the more important part of the structure.
	\end{definition}
	\begin{definition}[T-complex morphisms and the category of T-complexes]
		Let $(P,\phi)$ and $(Q,\psi)$ be two T-complexes. A natural transformation $\eta:P\to Q$ is called a \textit{T-complex morphism} if the following diagram, in particular the lower triangle, commutes for every centralized $C\in\ASC_n$ and $s:\X(C)\to P$:
		\[\begin{tikzcd}
			&&\X(C)\arrow{d}{s}\arrow[hook]{ld}[swap]{\iota(C)}\\
			&\Y(n)\arrow{r}{\phi^s}\arrow{dr}[swap]{\psi^{\eta\circ s}}&P\arrow{d}{\eta}\\
			&&Q
		\end{tikzcd}\]
		Summarized in an equation, it just says $\eta\circ\phi^s=\psi^{\eta\circ s}$, i.e. that $\eta$ preserves fillers.
		
		Clearly, every identity transformation is a T-complex morphism, as well as the composition of two T-complex morphisms. Therefore, the T-complexes and morphisms between them form a category, which shall be denoted by $\Tcplx$.
	\end{definition}
	\subsection{Examples for T-complexes and T-complex morphisms}
	Here we give a couple of important examples for T-complexes and T-complex morphisms that we will encounter again.
	
	First we introduce some related functors:
		\begin{definition}[Constant embedding]
		Let $\K:\Set\hookrightarrow\PSF$ denote the embedding that sends each set $A$ to the constant presheaf which simply sends every object in $\Fin_+$ to $A$ and every morphism to $\id_A$.
	\end{definition}
	It is easy to see that $\K(1)\cong\Y(1)$, and both of which are terminal objects in $\PSF$.
	\begin{definition}
		Let $\Q:\PSF\twoheadrightarrow\Set$ be the functor that sends each presheaf $P$ to the set $P(1)$ and each natural transformation $\eta:P\to Q$ to its first component $\eta_1:P(1)\to Q(1)$. By abuse of notation, we shall also sometimes take $\Tcplx$ to be the domain of $\Q$, i.e. $\Q:\Tcplx\twoheadrightarrow\Set$.
	\end{definition}
	\begin{definition}[Extended Yoneda-embedding]
		The domain of the Yoneda-embedding $\Y:\Fin_+\hookrightarrow\PSF$ can indeed be extended to $\Set$, i.e. giving rise to a functor $\Set\hookrightarrow\PSF$, which maps each set $A$ and each $n\in\N_+$ to the set $\Set([n],A)$. By abuse of notation, we shall also denote this functor by $\Y:\Set\hookrightarrow\PSF$, since there is no room for ambiguity. Yoneda-lemma still guarantees that it is fully faithful since $\Fin_+$ is a subcategory of $\Set$, and it is still an embedding since the following composition maps each set $A$ to $\Set([1], A)$, so it is naturally isomorphic to the identity functor through $\g$:
		\[\Set\xrightarrow\Y\PSF\xrightarrow\Q\Set\]
	\end{definition}
	\begin{theorem}\label{adjQPSF}
		The functor $\Q:\PSF\twoheadrightarrow\Set$ has both a left and a right adjoint. The left adjoint is $\K:\Set\to\PSF$, and the right adjoint is $\Y:\Set\to\PSF$.
	\end{theorem}
	\begin{proof}
		To show that $K$ is the left adjoint of $\Q$, we show that the following (obviously) natural transformation is componentwise bijective, hence a natural isomorphism:
		\begin{align*}
			\PSF(\K(A), P)&\to\Set(A, P(1))\\
			\eta&\mapsto\eta_1
		\end{align*}
		We do it by showing that for each set-theoretic function $f:A\to P(1)$, there is exactly one $\eta:\K(A)\to P$ s.t. $\eta_1=f$.
		Suppose that such an $\eta$ exists. By naturality, the following square must commute for every $n\in\N_+$:
		\[\begin{tikzcd}
			\K(A)(1)\arrow{r}{\K(A)(!_n)}\arrow{d}[swap]{\eta_1}&\K(A)(n)\arrow{d}{\eta_n}\\
			P(1)\arrow{r}[swap]{P(!_n)}&P(n)
		\end{tikzcd}\]
		However, since $\eta_1=f$ and the top row is simply the identity function $\id_A$ on $A$, it simplifies to the following commutative triangle:
		\[\begin{tikzcd}
			A\arrow{d}[swap]{f}\arrow{dr}{\eta_n}\\
			P(1)\arrow{r}[swap]{P(!_n)}&P(n)
		\end{tikzcd}\]
		Which reads $\eta_n=P(!_n)\circ f$ as an equation. So if $\eta:\K(A)\to P$ with $\eta_1=f$ exists, it must be uniquely defined by this equation, which, one can easily verify, indeed yields a valid natural transformation.
		
		To show that $\Y$ is the right adjoint of $\Q$, we show that the following (obviously) natural transformation is also componentwise bijective:
		\begin{align*}
			\Set(P(1),A)&\gets\PSF(P,\Y(A))\\
			\g_A^{-1}\circ\eta_1&\mapsfrom\eta
		\end{align*}
		We do it by showing that for each set-theoretic function $f:P(1)\to A$, there is exactly one $\eta:P\to\Y(A)$ s.t. $f=\g_A^{-1}\circ\eta_1$. Suppose that such an $\eta$ exists. By naturality, the following square must commute for every $n\in\N_+$ and $k\in[n]$:
		\[\begin{tikzcd}
			P(n)\arrow{rr}{P(\g_n(k))}\arrow{d}[swap]{\eta_n}&&P(1)\arrow{d}{\eta_1}\\
			\Set(n,A)\arrow{rr}[swap]{\Set(\g_n(k),\id_A)}&&\Set(1,A)
		\end{tikzcd}\]
		Now, take an arbitrary $x\in P(n)$, we can turn the commutative square into the following equation:
		\[\eta_n(x)\circ\g_n(k)=\eta_1(P(\g_n(k))(x))\]
		Both sides are now functions of type $1\to A$. Applying them to 0 and using the fact $\g_n(k)(0)=k$, we get
		\begin{equation}\label{pointwise}
			\eta_n(x)(k)=\eta_1(P(\g_n(k))(x))(0)
		\end{equation}
		The right-hand side is still a handful to write. However, remember we haven't used the assumption $f=\g_A^{-1}\circ\eta_1$ yet. Applying both sides of this equation to some arbitrary $y\in P(1)$, we get:
		\[f(y)=\g_A^{-1}(\eta_1(y))=\eta_1(y)(0)\]
		Setting $y=P(\g_n(k))(x)$ and putting it together with equation \ref{pointwise} above, we get
		\[\eta_n(x)(k)=f(P(\g_n(k))(x))\]
		This equality would uniquely determine $\eta:P\to\Y(A)$, and indeed, one may easily check that this gives a well-defined natural transformation.
	\end{proof}
	The first example has already been foreshadowed by Theorem \ref{lift}:
	\begin{theorem}[$\Y(A)$ are T-complexes]\label{YTcplx}
		For every set $A$, $\Y(A):\Fin_+^\op\to\Set$ can be given a T-complex structure in a unique way, such that any natural transformation from a T-complex $P$ into $\Y(A)$ is also a T-complex morphism.
	\end{theorem}
	\begin{proof}
		For the case of $A=[n]$, theorem \ref{lift} guarantees that every simplicial complex (even non-centralized ones) in $\Y(n)$ has a unique lifting simplex, which can be chosen as fillers. Furthermore, since lifting simplex are unique, both axioms in Definition \ref{Tcplx} are automatically satisfied, and for the same reason they are also preserved by every natural transformation into $\Y(n)$.
		
		The general case is completely analogous.
	\end{proof}
	\begin{lemma}[Centralized simplicial complexes in constant presheaves]\label{cscK}
		Let $c:\X(C)\to\K(A)$ be a centralized simplicial complex in $\K(A)$. Then there is some global element $\g_x:1\hookrightarrow A$ s.t. $c$ is equal to the following composition:
		\[\X(C)\xrightarrow{!_{\X(C)}}\K(1)\xrightarrow{\K(\g_x)}\K(A)\]
	\end{lemma}
	\begin{proof}
		Say $C\in\ASC_n$. Because $C$ is centralized, we have some $k\in Z(C)\subset[n]$. Let $x:=c_1(\g_k)\in A$.
		
		We show that $c_m(f)=x$ for every $f\in\X(C)(m)$, i.e. $c$ must in fact be constant, which is the same as saying that $c$ factors through $\K(\g_x)$. Consider the function $\langle\g_k;f\rangle:1+m\to n$, which maps 0 to $k$ and $1+j$ to $f(j)$ for $j\in[m]$. Furthermore, let $i^1:1\to1+m$ and $i^2:m\to1+m$ be injections into the sum $[1+m]$. Obviously per definition of sum/coproduct we have:
		\[\langle\g_k;f\rangle\circ i^1=\g_k\qquad\langle\g_k;f\rangle\circ i^2=f\]
		Furthermore, $\im\langle\g_k;f\rangle=\im(\g_k)\cup\im(f)=\{k\}\cup\im(f)\in C$, since $k\in Z(C)$ and $\im(f)\in C$. So $\langle\g_k;f\rangle\in\X(C)(1+m)$. In addition, regarding $i^1$, we have the following commutative square:
		\[\begin{tikzcd}
			\X(C)(1+m)\arrow{r}{\X(C)(i^1)}\arrow{d}[swap]{c_{1+m}}&X(C)(1)\arrow{d}{c_1}\\
			\K(A)(1+m)\arrow{r}[swap]{\K(A)(i^1)}&\K(A)(1)
		\end{tikzcd}\]
		However, the bottom row is simply the identity function $\id_A$ on $A$. So the diagram simplifies to:
		\[\begin{tikzcd}
			\X(C)(1+m)\arrow{r}{\X(C)(i^1)}\arrow{d}[swap]{c_{1+m}}&\X(C)(1)\arrow{dl}{c_1}\\
			A
		\end{tikzcd}\]
		In equation, it says $c_{1+m}=c_1\circ\X(C)(i^1)$. Specifically for $\langle\g_k;f\rangle\in\X(C)(1+m)$, it says
		\begin{equation}\label{triangle}
			c_{1+m}\langle\g_k;f\rangle=c_1(\langle\g_k;f\rangle\circ i^1)=c_1(\g_k)=x
		\end{equation}
		Similarly, regarding $i^2$, the following diagram also commutes:
		\[\begin{tikzcd}
			\X(C)(1+m)\arrow{r}{\X(C)(i^2)}\arrow{d}[swap]{c_{1+m}}&\X(C)(m)\arrow{dl}{c_m}\\
			A
		\end{tikzcd}\]
		In equation: $c_{1+m}=c_m\circ\X(C)(i^2)$, and specifically for $\langle\g_k;f\rangle$:
		\[c_{1+m}\langle\g_k;f\rangle=c_m(\langle\g_k;f\rangle\circ i^2)=c_m(f)\]
		Together with equation (\ref{triangle}) above, this gives $c_m(f)=x$, which is what we claimed at the beginning of the paragraph.
	\end{proof}
	\begin{theorem}[Constant presheaves are T-complexes]\label{TcplxK}
		For every set $A$, $\K(A)$ can be given a T-complex structure in a unique way. Furthermore, every natural transformation $\eta:P\to\K(A)$ from some T-complex $P$ into $\K(A)$, as well as every natural transformation $\varepsilon:\K(A)\to P$ from $\K(A)$ into a T-complex $P$, is a T-complex morphism.
	\end{theorem}
	\begin{proof}
		Consider a simplicial complex $c:\X(C)\to\K(A)$ with $C\in\ASC_n$, $n\in\N_+$ and suppose it has a lifting simplex $s:\Y(n)\to\K(A)$. By the previous Lemma \ref{cscK}, $c$ must factor through some $\K(\g_x)$ with $x\in A$, and $s$ must also factor through some $\K(\g_y)$ with $y\in A$. Consider the following diagram
		\[\begin{tikzcd}
			\X(C)\arrow{rr}{c}\arrow[two heads]{dr}{!_{\X(C)}}\arrow[bend right=30]{dddr}[swap]{\iota_C}&&\K(A)\\
			&\K(1)\arrow[hook,shift left=1mm]{ur}{\g_x}\arrow[hook',shift right=1mm]{ur}[swap]{\g_y}\\\\
			&\Y(n)\arrow[two heads]{uu}{!_{\Y(n)}}\arrow[bend right=30]{uuur}[swap]{s}
		\end{tikzcd}\]
		It's not hard to see that both paths $\X(C)\twoheadrightarrow\K(1)\hookrightarrow\K(A)$ in this diagram are equal, and that the unique natural transformation $!_{\X(C)}:\X(C)\twoheadrightarrow\K(1)$ is componentwise surjective ($\X(C)(m)$ is nonempty for every $m\in\N_+$) and hence indeed an epimorphism. So, it follows that $\g_x=\g_y$, thus $s=\g_x\circ\ !_{\Y(n)}$ is uniquely determined.
		
		So we've shown that every simplicial complex in $\K(A)$ has a unique lifting simplex, and by the same reasoning as Theorem \ref{YTcplx} it follows that $\K(A)$ can be given a T-complex structure in a unique way and that every natural transformation from some T-complex into $\K(A)$ is a T-complex morphism.
		
		To show that a natural transformation $\varepsilon:\K(A)\to P$ into some T-complex $P$ also preserves fillers, consider some simplicial complex $c:\X(C)\to\K(A)$ with $C\in\ASC_n$. We need to show the following diagram, in particular the right triangle, commutes:
		\[\begin{tikzcd}
			\X(C)\arrow{r}{c}\arrow{dr}[swap]{\iota_C}&\K(A)\arrow{r}{\varepsilon}&P\\
			&\Y(n)\arrow{u}\arrow{ur}[swap]{\phi^c}
		\end{tikzcd}\]
		Again, by invoking Lemma \ref{cscK}, we decompose $c$ into $\X(C)\twoheadrightarrow\K(1)\xrightarrow{\g_x}\K(A)$ with some $x\in A$, and by replacing $\K(1)$ with $\Y(1)$, we get the following diagram, which commutes because of (\ref{epifac}):
		\[\begin{tikzcd}
			\X(C)\arrow[two heads]{r}\arrow{dr}[swap]{\iota_C}&\Y(1)\arrow[hook]{r}&\K(A)\arrow{d}{\varepsilon}\\
			&\Y(n)\arrow[two heads]{u}[swap]{\Y(!_n)}\arrow{r}[swap]{\phi^c}&P
		\end{tikzcd}\]
	\end{proof}
	Because of Theorem \ref{YTcplx} and \ref{TcplxK}, we shall consider the functors $\Y$ and $\K$ to take values in $\Tcplx$ instead of $\PSF$, i.e. $\Y,\K:\Set\hookrightarrow\Tcplx$, since there is hardly any ambiguity.
	\begin{theorem}\label{adjQT}
		The restricted version of $Q$, i.e. $\Q:\Tcplx\twoheadrightarrow\Set$ also has both a left and a right adjoint, with $\K:\Set\hookrightarrow\Tcplx$ being the left adjoint and $\Y:\Set\hookrightarrow\Tcplx$ the right adjoint.
	\end{theorem}
	\begin{proof}
		We already know from Theorem \ref{adjQPSF} of the adjoints of $\Q:\PSF\to\Set$, and of the following two natural isomorphisms:
		\[\PSF(\K(A),P)\cong\Set(A,P(1))\qquad\Set(P(1),A)\cong\PSF(P,\Y(A))\]
		We also have the following equality between sets from Theorem \ref{YTcplx}:
		\[\PSF(P,\Y(A))=\Tcplx(P,\Y(A))\]
		as well as the following from Theorem \ref{TcplxK}:
		\[\PSF(\K(A),P)=\Tcplx(\K(A),P)\]
		whenever $P:\Fin_+^\op\to\Set$ is a T-complex. Putting them all together, we get the following natural isomorphisms:
		\[\Tcplx(\K(A),P)\cong\Set(A,P(1))\qquad\Set(P(1),A)\cong\Tcplx(P,\Y(A))\]
		which means $\Q:\Tcplx\twoheadrightarrow\Set$ has the same adjunction property as $\Q:\PSF\twoheadrightarrow\Set$.
	\end{proof}
	\begin{definition}[Global elements in $\PSF$ and $\Tcplx$]
		For $P:\Fin_+^\op\to\Set$ a presheaf and $x\in P(1)$, let $\kappa^x:\K(1)\to P$ denote the adjunct of the global element $\g_{P(1)}(x)$. To spell it out, each component $\kappa^x_n:1\to P(n)$ is defined as the following composition:
		\[1\xrightarrow{\g_{P(1)}(x)}P(1)\xrightarrow{P(!_n)}P(n)\]
		Furthermore, Theorem \ref{TcplxK} guarantees that $\kappa^x$ is a T-complex morphism whenever $P$ is a T-complex, and Theorems \ref{adjQPSF} and \ref{adjQT} assert that every global element in $\PSF$ as well as in $\Tcplx$ must be of this form.
	\end{definition}
	\section{$\Tcplx$ as a topos}
	\begin{comment}
	\subsection{Polyhedral T-complexes}
	\begin{definition}[Curried cartesian product]
		Let $\M:\Fin_+\to\Fin_+^{\Fin_+}$ be the adjunct/curried version of the cartesian product functor $\times:\Fin_+\times\Fin_+\to\Fin_+$.
	\end{definition}
	\begin{definition}[Polyhedral presheaves]
		Let $P:\Fin_+^\op\to\Set$ be a presheaf and $n\in\N_+$. Because $\Fin_+$ has binary products, $P\circ\M(n)$ is also a presheaf over $\Fin_+$. We call this the \textit{$n$-polyhedral presheaf} for $P$, or the \textit{presheaf of $n$-polyhedra} in $P$.
	\end{definition}
	The goal of this subsection is to prove that for every T-complex $(P,\phi)$ and every $n\in\N_+$, $P\circ\M(n)$ can also be given a T-complex structure.
	\begin{definition}\label{polyscomp}
		Let $m,n\in\N_+$, $C\in\ASC_n$ and $c:\X(C)\to P\circ\M(m)$ a simplicial complex in the polyhedral presheaf. Let $\pi^1_{m,n}:m\times n\to m$ and $\pi^2_{m,n}:m\times n\to n$ denote the canonical projections from the cartesian product in $\Fin_+$. Then for every $f:l\to m\times n$ with $\im(f)\in C^{\pi^2_{m,n}}$, we can obtain the following:
		\[f:l\to m\times n\quad\longmapsto\quad\pi^2_{m,n}\circ f:l\to n\quad\longmapsto\quad c_l(\pi^2_{m,n}\circ f)\in P(m\times l)\]
		as well as:
		\[f:l\to m\times n\longmapsto(\pi^1_{m,n}\circ f;\ \id_l):l\to m\times l\longmapsto P(\pi^1_{m,n}\circ f;\ \id_l):P(m\times l)\to P(l)\]
		Putting both together we get an element of $P(l)$. We shall name this operation $\tau(c)_l$ for each $l\in\N_+$. To summarize, $\tau(c)_l$ is defined as follows:
		\begin{equation}\label{tau}
			\tau(c)_l:\X(C^{\pi^2_{m,n}})(l)\to P(l)\qquad f\mapsto P(\pi^1_{m,n}\circ f;\id_l)(c_l(\pi^2_{m,n}\circ f))
		\end{equation}
	\end{definition}
	\begin{theorem}
		For every simplicial complex $c:\X(C)\to P\circ\M(m)$, $\tau(c)_l:\X(C^{\pi^2_{m,n}})(l)\to P(l)$ is natural in $l\in\Fin_+$. Hence, we may view $\tau(c):\X(C^{\pi^2_{m,n}})\to P$ as a natural transformation and hence a simplicial complex in $P$.
	\end{theorem}
	% TODO: rewrite using diagrams and dinatural transformations.
	\begin{proof}
		We simply have to prove that for every $g:k\to l$, the following diagram commutes:
		\[\begin{tikzcd}
			\X(C^{\pi^2_{m,n}})(l)\arrow{r}{-\circ g}\arrow{d}[swap]{\tau(c)_l}&\X(C^{\pi^2_{m,n}})(k)\arrow{d}{\tau(c)_k}\\
			P(l)\arrow{r}[swap]{P(g)}&P(k)
		\end{tikzcd}\]
		So consider an arbitrary $f\in\X(C^{\pi^2_{m,n}})(l)$. We have:
		\begin{align*}
			&\tau(c)_k(f\circ g)\\
			=\ &P(\pi^1_{m,n}\circ f\circ g;\ \id_k)(c_k(\pi^2_{m,n}\circ f\circ g))\\
			=\ &P(\pi^1_{m,n}\circ f\circ g;\ \id_k)(P(\id_m\times g)(c_l(\pi^2_{m,n}\circ f)))\\
			=\ &P((\id_m\times g)\circ(\pi^1_{m,n}\circ f\circ g;\ \id_k))(c_l(\pi^2_{m,n}\circ f))\\
			=\ &P(\pi^1_{m,n}\circ f\circ g;g)(c_l(\pi^2_{m,n}\circ f))\\
			=\ &P((\pi^1_{m,n}\circ f;\ \id_l)\circ g)(c_l(\pi^2_{m,n}\circ f))\\
			=\ &P(g)(P(\pi^1_{m,n}\circ f;\id_l)(c_l(\pi^2_{m,n}\circ f)))\\
			=\ &P(g)(\tau(c)_l)
		\end{align*}
	\end{proof}
	\begin{remark}
		I am sure there is a more visual proof of the above theorem using commutative diagrams rather than a nightmare of parentheses, that would however require familiarity with dinatural transformations in order to write or understand. As I am not familiar with them, I'll have to stick to equations for now. If you're unfortunately reading this before I come up with a more visual proof, I am sorry.
	\end{remark}
	So, for every simplicial complex $c:\X(C)\to P\circ\M(m)$ of shape $C$ in the $n$-polyhedral presheaf, there is a corresponding simplicial complex $\tau(c):\X(C^{\pi^2_{m,n}})\to P$ of shape $C^{\pi^2_{m,n}}$ in the original presheaf $P$. The following theorem states that this is indeed a one-to-one correspondence, i.e. a bijection:
	\begin{theorem}\label{taubij}
		The function $\PSF(\X(C),P\circ\M(m))\to\PSF(\X(C^{\pi^2_{m,n}}),P)$, $c\mapsto\tau(c)$ is bijective, with inverse given by $s\mapsto\sigma(s)$, where $\sigma(s)_l$ for each $l\in\N_+$ is defined by the following obviously natural mapping:
		\[f:l\to n\quad\longmapsto\quad\id_m\times f:m\times l\to m\times n\quad\longmapsto\quad s_{m\times l}(\id_m\times f)\in P(m\times l)\]
		i.e.
		\begin{equation}\label{sigma}
			\sigma(s)_l:\X(C)(l)\to P(m\times l),\qquad f\mapsto s_{m\times l}(\id_m\times f)
		\end{equation}
	\end{theorem}
	\begin{proof}
		First of all, for every $f:l\to n$ with $\im(f)\in C$, we have
		\[\pi^2_{m,n}[\im(\id_m\times f)]=\im(\pi^2_{m,n}\circ(\id_m\times f))=\im(f\circ\pi^2_{m,l})=\im(f)\in C\]
		so $\im(\id_m\times f)\in C^{\pi^2_{m,n}}$, hence the mapping in \ref{sigma} is well-defined.
		Now, we show that for every simplicial complex $c:\X(C)\to P\circ\M(m)$, $\sigma(\tau(c))=c$. For each $l\in\N_+$ and $f\in\X(C)(l)$, we have:
		\begin{align*}
			\sigma(\tau(c))_l(f)&=\tau(c)_{m\times l}(\id_m\times f)\\
			&=P(\pi^1_{m,n}\circ(\id_m\times f);\id_{m\times l})(c_{m\times l}(\pi^2_{m,n}\circ(\id_m\times f)))\\
			&=P(\pi^1_{m,l};\id_{m\times l})(c_{m\times l}(f\circ\pi^2_{m,l}))\\
			&=P(\pi^1_{m,l};\id_{m\times l})(P(\id_m\times\pi^2_{m,l})(c_l(f)))\\
			&=P((\id_m\times\pi^2_{m,l})\circ(\pi^1_{m,l};\id_{m\times l}))(c_l(f))\\
			&=P(\pi^1_{m,l};\pi^2_{m,l})(c_l(f))\\
			&=P(\id_{m\times l})(c_l(f))\\
			&=c_l(f)
		\end{align*}
		Hence $\sigma(\tau(c))=c$. Now conversely, for every $s:\X(C^{\pi^2_{m,n}})\to P$ and each $l\in\N_+$ and $f\in\X(C^{\pi^2_{m,n}})(l)$, we have
		\begin{align*}
			\tau(\sigma(s))_l(f)&=P(\pi^1_{m,n}\circ f;\id_l)(\sigma(s)_l(\pi^2_{m,n}\circ f))\\
			&=P(\pi^1_{m,n}\circ f;\id_l)(s_{m\times l}(\id_m\times(\pi^2_{m,n}\circ f)))\\
			&=s_l((\id_m\times(\pi^2_{m,n}\circ f))\circ(\pi^1_{m,n}\circ f;\id_l))\\
			&=s_l(\pi^1_{m,n}\circ f;\pi^2_{m,n}\circ f)\\
			&=s_l((\pi^1_{m,n};\pi^2_{m,n})\circ f)\\
			&=s_l(\id_{m\times n}\circ f)\\
			&=s_l(f)
		\end{align*}
		Hence $\tau(\sigma(s))=s$.
	\end{proof}
	\begin{remark}
		Again, I sincerely apologize for my current inability to find a more elegant and visual proof than this nightmare of parentheses. To whoever is reading this, I hope your head doesn't explode; if it does, however, I hope then that you'll find an elegant proof in heaven.
	\end{remark}
	\begin{lemma}\label{sigstruct}
		$\sigma$ and $\tau$ preserve certain structures between simplicial complexes, summarized in the following two points:
		\begin{enumerate}
			\item Given a simplex $s:\Y(m\times n)\to P$ and $C\in\ASC_n$, $\sigma$ maps the following composition:
			\[\X(C^{\pi^2_{m,n}})\xrightarrow{\iota_{C^{\pi^2}}}\Y(m\times n)\xrightarrow{s}P\]
			to the following composition:
			\[\X(C)\xrightarrow{\iota_C}\Y(n)\xrightarrow{\sigma(s)}P\circ\M(m)\]
			\item Given $C\in\ASC_n$, $c:\X(C^{\pi^2_{m,n}})\to P$ and $f:k\to n$, $\sigma$ maps the following composition:
			\begin{align*}
				\X((C^f)^{\pi^2_{m,k}})=\X(C^{f\circ\pi^2_{m,k}})=\X(C^{\pi^2_{m,n}\circ(\id_m\times f)})=\X((C^{\pi^2_{m,n}})^{\id_m\times f})\\
				\xrightarrow{\X(\id_m\times f)}\X(C^{\pi^2_{m,n}})\xrightarrow{c}P
			\end{align*}
			to:
			\[\X(C^f)\xrightarrow{\X(f)}\X(C)\xrightarrow{\sigma(c)}P\circ\M(m)\]
		\end{enumerate}
		And, of course, since $\tau$ is the inverse of $\sigma$, it maps the second composition in each case to the first.
	\end{lemma}
	\begin{proof}
		\begin{enumerate}
			\item Trivial, since composition with $\iota_C$ or $\iota_{C^{\pi^2}}$ is just the restriction to $\X(C)$ or $\X(C^{\pi^2})$, respectively. And from the defining equation of $\sigma(s)$, we can see restricting its domain has the same effect as restricting the domain of $s$, before applying $\sigma$ to it.
			\item Applying $\sigma$ to the upper composition, we get
			\[\sigma(c\circ\X(\id_m\times f))\]
			Applying this to an arbitrary $g\in\X(C^f)(l)$, i.e. $g:l\to k$ with $\im(f\circ g)\in C$, we get
			\begin{align*}
				&\sigma(c\circ\X(\id_m\times f))_l(g)=c_{k\times l}(\X(\id_k\times f)_{k\times l}(\id_k\times g))\\
				=\ &c_{k\times l}((\id_k\times f)\circ(\id_k\times g))=c_{k\times l}(\id_k\times(f\circ g))\\
				=\ &\sigma(c)_l(f\circ g)=\sigma(c)_l(\X(f)_l(g))
			\end{align*}
			where the last expression is exactly the lower composition applied to $g$.
		\end{enumerate}
	\end{proof}
	The lemma paves way for the following theorem, which is the focus of the subsection:
	\begin{theorem}[$m$-polyhedral T-complex]\label{polyT}
		Given a T-complex $(P,\phi)$ and an $m\in\N$, the $m$-polyhedral presheaf $P\circ\M(m)$ can also be made into a T-complex $(P\circ\M(m),\psi)$, with $\psi^c:=\sigma(\phi^{\tau(c)})$, i.e. it simply transforms a simplicial complex $c$ in $P\circ\M(m)$ into one in $P$ using $\tau$, fills it using $\phi$, then transforms the filled simplex in $P$ into a simplex in $P\circ\M(m)$ again using $\sigma$.
	\end{theorem}
	\begin{proof}
		First, we show that $\psi^c:\Y(n)\to P\circ\M(m)$ indeed lifts every $c:\X(C)\to P\circ\M(m)$ for $C\in\ASC_n$, i.e. the following triangle commutes:
		\[\begin{tikzcd}
			\X(C)\arrow{r}{c}\arrow[hook]{d}[swap]{\iota_C}&P\circ\M(m)\\
			\Y(n)\arrow{ur}[swap]{\psi^c=\sigma(\phi^{\tau(c)})}
		\end{tikzcd}\]
		According to Lemma \ref{sigstruct}, it can be obtained by applying $\sigma$ to the following equivalent triangle:
		\[\begin{tikzcd}
			\X(C^{\pi^2})\arrow{r}{\tau(c)}\arrow[hook]{d}[swap]{\iota_{C^{\pi^2}}}&P\\
			\Y(m\times n)\arrow{ur}[swap]{\phi^{\tau(c)}}
		\end{tikzcd}\]
		which obviously commutes since $(P,\phi)$ is a T-complex and hence $\phi^{\tau(c)}$ lifts $\tau(c)$.
		Now, we prove the first axiom, i.e. that the following diagram commutes for centralized $C\in\ASC_n$ and $f:k\to n$ s.t. $C^f$ is also centralized:
		\[\begin{tikzcd}
			\X(C^f)\arrow{rr}{\X(f)}\arrow[hook]{dd}[swap]{\iota_{C^f}}&&\X(C)\arrow{dd}{c}\arrow[hook]{ld}[swap]{\iota_C}\\
			&\color{red}{\Y(n)}\arrow[color=red]{dr}[swap]{\psi^c=\sigma(\phi^{\tau(c)})}\\
			\color{red}{\Y(k)}\arrow[color=red]{ur}{\Y(f)}\arrow[color=red]{rr}[swap]{\psi^{c\circ\X(f)}=\sigma(\phi^{\tau(c\circ\X(f))})}&&\color{red}{P\circ\M(m)}
		\end{tikzcd}\]
		The only important part is of course the lower triangle, which is shown here in red. Again, according to Lemma \ref{sigstruct} and Theorem \ref{taubij}, it can be obtained by applying $\sigma$ to the red triangle shown in the following diagram:
		\[\begin{tikzcd}
			\X((C^f)^{\pi^2})=\X((C^{\pi^2})^{\id_m\times f})\arrow[hook]{dd}[swap]{\iota}\arrow{rr}{\X(\id_m\times f)}&&\X(C^{\pi^2})\arrow[hook]{ld}[swap]{\iota_{C^{\pi^2}}}\arrow{dd}{\tau(c)}\\
			&\color{red}{\Y(m\times n)}\arrow[color=red]{dr}[swap]{\phi^{\tau(c)}}\\
			\color{red}{\Y(m\times k)}\arrow[color=red]{ur}{\Y(\id_m\times f)}\arrow[color=red]{rr}[swap]{\phi^{\tau(c)\circ\X(\id_m\times f)}}&&\color{red}P
		\end{tikzcd}\]
		which again obviously commutes since $(C^f)^{\pi^2}$ is also centralized according to Corollary \ref{centerpi}, and $(P,\phi)$ is a T-complex and hence satisfies axiom (\ref{pullbackcomp}).
		
		Now, to prove the second axiom, suppose we have $c:\X(C)\twoheadrightarrow\Y(k)$ epic and centralized, and a $k$-simplex $s:\Y(k)\to P\circ\M(m)$ in the $m$-polyhedral presheaf:
		\[\X(C)\xrightarrow{c}\Y(k)\xrightarrow{s}P\circ\M(m)\]
		We then need to show that (\ref{epifac}) commutes, i.e. $\phi^{s\circ c}=s\circ\F(c):\Y(n)\to\Y(k)$. First, we can factorize $c$ further according to Theorem \ref{lift}:
		\[\X(C)\xrightarrow{\iota_C}\Y(n)\xrightarrow{\F(c)}\Y(k)\xrightarrow{s}P\circ\M(m)\]
		Applying $\tau$ to it, we get the following composition according to Lemma \ref{sigstruct} and Theorem \ref{taubij}:
		\[\X(C^{\pi^2})\xrightarrow{\iota_{C^{\pi^2}}}\Y(m\times n)\xrightarrow{\id_m\times\F(c)}\Y(m\times k)\xrightarrow{\tau(s)}P\]
		Now, let $d:\X(C^{\pi^2})\to\Y(m\times k)$ denote the composition of the first two arrows. So far, what we've proven is simply $\tau(s\circ c)=\tau(s)\circ d$, but also $\F(d)=\id_m\times\F(c)$ as the simplex $\Y(m\times n)\to\Y(m\times k)$ that lifts $d$ is unique. Furthermore, we see that $d$ is also epic, because there is the following situation with $\eta\circ d=\varepsilon\circ d$ for some presheaf $Q$ and some $\eta,\varepsilon:\Y(m\times k)\to Q$:
		\[\begin{tikzcd}
			\X(C^{\pi^2})\arrow{r}{d}&\Y(m\times k)\arrow[shift left=1mm]{r}{\eta}\arrow[shift right=1mm]{r}[swap]{\varepsilon}&Q
		\end{tikzcd}\]
		Applying $\sigma$ to it yields the following with $\sigma(\eta)\circ c=\sigma(\varepsilon)\circ c$:
		\[\begin{tikzcd}
			\X(C)\arrow[two heads]{r}{c}&\Y(k)\arrow[shift left=1mm]{r}{\sigma(\eta)}\arrow[shift right=1mm]{r}[swap]{\sigma(\varepsilon)}&Q\circ\M(m)
		\end{tikzcd}\]
		Since $c$ is epic, we have $\sigma(\eta)=\sigma(\varepsilon)$, and further applying $\tau$ to this equation gives $\eta=\varepsilon$.
		
		Hence, since $(P,\phi)$ is a T-complex and $d:\X(C^{\pi^2})\twoheadrightarrow\Y(m\times k)$ is epic, we have the following commutative triangle:
		\[\begin{tikzcd}
			\Y(m\times k)\arrow{r}{\tau(s)}&P\\
			\Y(m\times n)\arrow{u}{\F(d)=\id_m\times\F(c)}\arrow{ur}[swap]{\phi^{\tau(s\circ c)}}
		\end{tikzcd}\]
		Finally, applying $\sigma$ and Lemma \ref{sigstruct} to this triangle gives us what we needed to show:
		\[\begin{tikzcd}
			\Y(k)\arrow{r}{s}&P\circ\M(m)\\
			\Y(n)\arrow{u}{\F(c)}\arrow{ur}[swap]{\sigma(\phi^{\tau(s\circ c)})=\psi^{s\circ c}}
		\end{tikzcd}\]
		i.e. $\psi^{s\circ c} = s\circ \F(c)$.
	\end{proof}
	To conclude this section, we take a look at a special class of natural transformations between polyhedral T-complexes, namely those of the form $\id_P*\M(f):P\circ\M(m)\to P\circ\M(l)$, where $*$ is the Godement product, i.e. the horizontal composition of the identity transformation $\id_P:P\to P$ and $\M(f):\M(l)\to\M(m)$ with $f:l\to m$. The goal is to show that these are indeed T-complex morphisms.
	\begin{lemma}\label{sigexch}
		Let $P:\Fin^\op\to Set$ be a presheaf, $f:l\to m$ a function, $c:\X(C^{\pi^2_{m,n}})\to P$ be a simplicial complex, with $C\in\ASC_n$. $\sigma$ then maps the following composition:
		\[\X(C^{\pi^2_{l,n}})=\X((C^{\pi^2_{m,n}})^{f\times\id_n})\xrightarrow{\X(f\times\id_n)}\X(C^{\pi^2_{m,n}})\xrightarrow{c}P\]
		to the following:
		\[\X(C)\xrightarrow{\sigma(c)}P\circ\M(m)\xrightarrow{\id_P*\M(f)}P\circ\M(l)\]
		And conversely, since $\tau$ is the inverse of $\sigma$, it maps the second composition to the first.
	\end{lemma}
	\begin{proof}
		Let $g:k\to n$ with $\im(g)\in C$. Apply $\sigma$ to the upper composition, then apply the result to $g$, we get:
		\begin{align*}
			&\sigma(c\circ\X(f\times\id_n))_k(g)=c_{l\times k}(\X(f\times\id_n)_{l\times k}(\id_l\times g))\\
			=\ &c_{l\times k}((f\times\id_n)\circ(\id_l\times g))=c_{l\times k}(f\times g)
		\end{align*}
		Apply the lower composition to $g$, we have:
		\begin{align*}
			&(\id_P*\M(f))_k(\sigma(c)_k(g))=P(f\times\id_k)(c_{m\times k}(\id_m\times g))\\
			=\ &c_{l\times k}(\X(C^{\pi^2_{m,n}})(f\times\id_k)(\id_m\times g))=c_{l\times k}((\id_m\times g)\circ(f\times\id_k))\\
			=\ &c_{l\times k}(f\times g)
		\end{align*}
		Hence the two are one and the same.
	\end{proof}
	Now, the promised theorem to conclude the section:
	\begin{theorem}\label{GodementThom}
		Let $(P,\phi)$ be a T-complex and $f:l\to m$ a function between canonical finite sets. Then $\id_P*\M(f):P\circ\M(m)\to P\circ\M(l)$ is a homomorphism between T-complexes. To spell it out: Let $\psi_m$ be the filler of $P\circ\M(m)$ and $\psi_l$ the filler of $P\circ\M(l)$. For every centralized simplicial complex $c:\X(C)\to P\circ\M(m)$, the following diagram commutes:
		\[\begin{tikzcd}
			\X(C)\arrow[hook]{dr}[swap]{\iota_C}\arrow{r}{c}&P\circ\M(m)\arrow{rr}{\id_P*\M(f)}&&P\circ\M(l)\\
			&\Y(n)\arrow{u}{\psi_m^c}\arrow{urr}[swap]{\psi_l^{(\id_P*\M(f))\circ c}}
		\end{tikzcd}\]
	\end{theorem}
	\begin{proof}
		The only important part of the diagram above that we need to show is the triangle on the right, here colored in red:
		\[\begin{tikzcd}
			\X(C)\arrow[hook]{dr}[swap]{\iota_C}\arrow{r}{c}&\color{red}{P\circ\M(m)}\arrow[color=red]{r}{\id_P*\M(f)}&\color{red}{P\circ\M(l)}\\
			&\color{red}{\Y(n)}\arrow[color=red]{u}{\psi_m^c}\arrow[color=red]{ur}[swap]{\psi_l^{(\id_P*\M(f))\circ c}}
		\end{tikzcd}\]
		This however is easily obtained by applying $\sigma$ and the above Lemma \ref{sigexch} to the red triangle in the following diagram (note that we have $\psi_m^c=\sigma(\phi^{\tau(c)})$ by definition, and similarly $\psi_l^{(\id_P*\M(f))\circ c}=\sigma(\phi^{\tau((\id_P*\M(f))\circ c)})=\sigma(\phi^{\tau(c)\circ\X(f\times\id_n)})$ from Lemma \ref{sigexch} again):
		\[\begin{tikzcd}
			\X(C^{\pi^2_{l,n}})=\X((C^{\pi^2_{m,n}})^{f\times\id_n})
			\arrow[hook]{dd}[swap]{\iota}
			\arrow{rr}{\X(f\times\id_n)}
			&&
			\X(C^{\pi^2_{m,n}})
			\arrow[hook]{ld}{\iota}
			\arrow{dd}{\tau(c)}
			\\&
			\color{red}{\Y(m\times n)}
			\arrow[color=red]{dr}{\phi^{\tau(c)}}
			\\
			\color{red}{\Y(l\times n)}
			\arrow[color=red]{ur}{\Y(f\times\id_n)}
			\arrow[color=red]{rr}[swap]{\phi^{\tau(c)\circ\X(f\times\id_n)}}
			&&
			\color{red}P
		\end{tikzcd}\]
		This diagram commutes because $C^{\pi^2_{l,n}}$ is centralized according to Corollary \ref{centerpi} and $(P,\phi)$ satisfies axiom (\ref{pullbackcomp}).
	\end{proof}
	Another subsection later, we will be defining the internal hom-functor in $\Tcplx$ in terms of polyhedral T-complexes, and thus show that $\Tcplx$ is cartesian closed.
	\end{comment}
	\subsection{Products and Sums}
	In this subsection, we show that an arbitrary (small) product or sum of T-complexes is again a T-complex. This means the category $\Tcplx$ has all (small) products and sums.
	
	The case for products is quite straightforward:
	\begin{theorem}[Product of T-complexes]
		Let $(P_i,\phi_i)_{i\in I}$ be a family of T-complexes. The product of there underlying presheaves, i.e. $\prod_iP_i$, can also be made into a T-complex in a unique way s.t. all projections $\pi_i:\prod_iP_i\to P_i$ are T-complex morphisms.
	\end{theorem}
	\begin{proof}
		Let $\prod_i\phi_i$ denote the fillers for $\prod_iP_i$ that we shall define. In order for all projections to be T-complex morphisms, it must satisfy the equation
		\begin{equation}\label{prodfill}
			\pi_i\circ\left(\prod_i\phi_i\right)^c=\phi_i^{\pi_i\circ c}
		\end{equation}
		for every centralized simplicial complex $c:\X(C)\to\prod_iP_i$ and $i\in I$. This indeed uniquely defines $\prod_i\phi_i$ because of the universal property of cartesian products.
		
		The fact that the fillers defined as such indeed satisfies the axioms in Definition \ref{Tcplx} also follows directly from the universal property of cartesian products, and that all $\phi_i$ satisfy these axioms. The exact proof shall hence not be spelled out here, and instead left as an exercise for the inquisitive among the readers.
	\end{proof}
	The same statement for sums is somewhat more complicated to prove and requires a lemma first:
	\begin{lemma}[Centralized simplicial complexes in sums of presheaves]\label{cscsum}
		Let $(P_i)_{i\in I}$ be a family of $\Fin_+$-presheaves, and $c:\X(C)\to\coprod_iP_i$ be a centralized simplicial complex into their sum. Then there exists some unique $i\in I$ and some $d:\X(C)\to P_i$ s.t. $c$ is equal to the composition of $d$ and the inclusion $P_i\hookrightarrow\coprod_iP_i$.
	\end{lemma}
	\begin{proof}
		Say $C\in\ASC_n$. Because $C$ is centralized, there is some $k\in Z(C)\subset[n]$. We have $c_1(\g_k)\in\coprod_iP_i(1)$, so there exists some unique $i$ with $c_1(\g_k)\in P_i(1)$.
		
		We now show that $c_m(f)\in P_i(m)$ for every $m\in\N_+$ and $f\in\X(C)(m)$, which is same as saying that $c$ factors through some $d:\X(C)\to P_i$. Similar to the proof of Lemma \ref{cscK}, we consider the function $\langle\g_k;f\rangle:1+m\to n$, and the inclusions $i^1:1\to1+m$ and $i^2:m\to1+m$. Per definition of sum/coproduct we have:
		\[\langle\g_k;f\rangle\circ i^1=\g_k\quad\langle\g_k;f\rangle\circ i^2=f\]
		and also $\im\langle\g_k;f\rangle=\im(\g_k)\cup\im(f)=\{k\}\cup\im(f)\in C$, as $k\in Z(C)$ and $\im(f)\in C$. By naturality of $c$, we have the following commutative diagram:
		\[\begin{tikzcd}
			\X(C)(1+m)\arrow{rr}{\X(C)(i^1)}\arrow{d}[swap]{c_{1+m}}&&
			\X(C)(1)\arrow{d}{c_1}\\
			(\coprod_iP_i)(1+m)\arrow{rr}[swap]{(\coprod_iP_i)(i^1)}&&
			(\coprod_iP_i)(1)
		\end{tikzcd}\]
		as well as
		\[\begin{tikzcd}
			\X(C)(1+m)\arrow{rr}{\X(C)(i^2)}\arrow{d}[swap]{c_{1+m}}&&
			\X(C)(m)\arrow{d}{c_m}\\
			(\coprod_iP_i)(1+m)\arrow{rr}[swap]{(\coprod_iP_i)(i^2)}&&
			(\coprod_iP_i)(m)
		\end{tikzcd}\]
		Applying both diagrams to $\langle\g_k;f\rangle\in\X(C)(1+m)$, we first have
		\begin{equation}\label{sumsquare1}
			\left(\coprod_iP_i\right)(i^1)(c_{1+m}\langle\g_k;f\rangle)=c_1(\langle\g_k;f\rangle\circ i^1)=c_1(\g_k)
		\end{equation}
		and then
		\begin{equation}\label{sumsquare2}
			\left(\coprod_iP_i\right)(i^2)(c_{1+m}\langle\g_k;f\rangle)=c_m(\langle\g_k;f\rangle\circ i^2)=c_m(f)
		\end{equation}
		Now recall $c_1(\g_k)\in P_i(1)$. Together with equation (\ref{sumsquare1}), this gives
		\[\left(\coprod_iP_i\right)(i^1)(c_{1+m}\langle\g_k;f\rangle)\in P_i(1)\]
		which in particular means $c_{1+m}\langle\g_k;f\rangle\in P_i(1+m)$. Plugging this into equation (\ref{sumsquare2}), we get
		\[c_m(f)=\left(\coprod_iP_i\right)(i^2)(c_{1+m}\langle\g_k;f\rangle)=P_i(i^2)(c_{1+m}\langle\g_k;f\rangle)\in P_i(m)\]
		which is what we claimed.
	\end{proof}
	With this, it is now easy to see how we can equip $\coprod_iP_i$ with a T-complex structure:
	\begin{theorem}\label{Tcplxsum}
		For every family of T-complexes $(P_i,\phi_i)_{i\in I}$, there is a unique way to equip their sum $\coprod_iP_i$ with a T-complex structure s.t. all inclusions $P_i\hookrightarrow\coprod_iP_i$ are T-complex homomorphisms.
	\end{theorem}
	\begin{proof}
		Again, let $\coprod_i\phi_i$ denote the fillers for $\coprod_iP_i$, and let $c:\X(C)\to\coprod_iP_i$ be an arbitrary centralized simplicial complex in the sum, with $C\in\ASC_n$. As proven in the previous Lemma \ref{cscsum}, it factors through some simplicial complex $d:\X(C)\to P_i$ for some unique $i\in I$. Hence, in order for the inclusion $P_i\hookrightarrow\coprod_iP_i$ to be a T-complex homomorphism, $\coprod_i\phi_i$ must be defined in a way s.t. the following diagram commutes:
		\[\begin{tikzcd}
			\X(C)\arrow{r}{d}\arrow{dr}[swap]{\iota_C}&P_i\arrow[hook]{r}&\coprod_iP_i\\
			&\Y(n)\arrow{u}{\phi_i^d}\arrow{ur}[swap]{\left(\coprod_i\phi_i\right)^c}
		\end{tikzcd}\]
		Indeed, it is again easy to check that this in fact uniquely defines $\coprod_i\phi_i$ and that it satisfies all the required axioms, thus making $\coprod_iP_i$ into a T-complex.
	\end{proof}
	\begin{remark}
		For a set $A$, we can view the constant presheaf $\K(A)$ as the $A$-indexed sum $\coprod_{x\in A}\Y(1)$. Thus, Lemma \ref{cscK} and Theorem \ref{TcplxK} (including their proofs) are simply special cases of Lemma \ref{cscsum} and \ref{Tcplxsum}
	\end{remark}
	\subsection{Exponentials}
	In order to define exponentials (a.k.a. internal homs) in $\Tcplx$, we start by revisiting a known construction for exponentials in $\PSF$:
	\begin{comment}
	\begin{theorem}\label{PSFCCC}
		For $P,Q:\Fin_+^\op\to\Set$, define the exponential $Q^P:\Fin_+^\op\to\Set$ as follows
		\begin{align*}
			Q^P(n)&:=\PSF(P,Q\circ\M(n))\\
			Q^P(f)&:Q^P(n)\to Q^P(m)\quad(\mathrm{for }\ f:m\to n)\\
			&\quad\qquad\eta\mapsto(\id_Q*\M(f))\circ\eta
		\end{align*}
		This makes $\PSF$ into a cartesian closed category. In particular, we have the following isomorphism that is natural in $P,Q,R$:
		\[\iota_{P,Q,R}:\PSF(P\times Q, R)\xrightarrow{\sim}\PSF(P,R^Q)\]
	\end{theorem}
	\begin{proof}
		We start by defining $\iota$. Expanding the definition step by step, here's what we need to define
		\begin{align*}
			\iota_{P,Q,R}&:\PSF(P\times Q, R)\to\PSF(P, R^Q)\\
			\iota_{P,Q,R}(\eta)&:P\to R^Q\\
			\iota_{P,Q,R}(\eta)_m&:P(m)\to R^Q(m)=\PSF(Q, R\circ\M(m))\\
			\iota_{P,Q,R}(\eta)_m(x)&:Q\to R\circ\M(m)\\
			\iota_{P,Q,R}(\eta)_m(x)_n&:Q(n)\to R(m\times n)
			\intertext{With $\eta:P\times Q\to R$, $m,n\in\N_+$ and $x\in P(m)$ arbitrary. We can now give an explicit expression for the last function:}
			&y\mapsto\eta_{m\times n}(P(\pi^1_{m,n})(x), Q(\pi^2_{m,n})(y))
		\end{align*}
		As it is rather tedious and boring to check that this is indeed a well-defined natural transformation, it is left as an exercise for the curious (or simply bored) reader. To show that this is an isomorphism, we give an explicit definition for its inverse, which we shall call $\kappa$. To define it we again expand the definition step by step:
		\begin{align*}
			\kappa_{P,Q,R}&:\PSF(P, R^Q)\to\PSF(P\times Q, R)\\
			\kappa_{P,Q,R}(\varepsilon)&:P\times Q\to R\\
			\kappa_{P,Q,R}(\varepsilon)_n&:P(n)\times Q(n)\to R(n)\\
			\intertext{With $\varepsilon:P\to Q^R$ and $n\in\N_+$ arbitrary. We can again give an explicit expression for the last function:}
			&(x,y)\mapsto R(\Delta_n)(\varepsilon_n(x)_n(y))
		\end{align*}
		Where $\Delta_n:=(\id_n;\id_n):n\to n\times n$ is the diagonal map. Again, it is left as an exercise for the reader to check that this is well-defined. We will however verify here that $\iota$ and $\kappa$ are indeed inverse to each other. For $\eta:P\times Q\to R$, $n\in\N_+$, $x\in P(n)$ and $y\in Q(n)$, we have
		\begin{align*}
			&\kappa_{P,Q,R}(\iota_{P,Q,R}(\eta))_n(x, y)\\
			=\ &R(\Delta_n)(\iota_{P,Q,R}(\eta)_n(x)_n(y))\\
			=\ &R(\Delta_n)(\eta_{n\times n}(P(\pi^1_{n,n})(x), Q(\pi^2_{n,n})(y)))\\
			=\ &\eta_n((P\times Q)(\Delta_n)(P(\pi^1_{n,n})(x), Q(\pi^2_{n,n})(y)))\\
			=\ &\eta_n(P(\pi^1_{n,n}\circ\Delta_n)(x),Q(\pi^2_{n,n}\circ\Delta_n)(x))\\
			=\ &\eta_n(P(\id_n)(x), Q(\id_n)(y))\\
			=\ &\eta_n(x,y)
		\end{align*}
		So $\kappa\circ\iota=\id$. For the other composition, take $\varepsilon:P\to R^Q$, $m,n\in\N_+$, $x\in P(m)$ and $y\in Q(n)$:
		\begin{align*}
			&\iota_{P,Q,R}(\kappa_{P,Q,R}(\varepsilon))_m(x)_n(y)\\
			=\ &\kappa_{P,Q,R}(\varepsilon)_{m\times n}(P(\pi^1_{m,n})(x), Q(\pi^2_{m,n})(y))\\
			=\ &R(\Delta_{m\times n})(\varepsilon_{m\times n}(P(\pi^1_{m,n})(x))_{m\times n}(Q(\pi^2_{m,n})(y)))\\
			=\ &R(\Delta_{m\times n})((\id_R*\M(\pi^1_{m,n}))(\varepsilon_m(x)_{m\times n}(Q(\pi^2_{m,n})(y))))\\
			=\ &R(\Delta_{m\times n})((\id_R*\M(\pi^1_{m,n}))(R(\id_m\times \pi^2_{m, n})(\varepsilon_m(x)_n(y))))\\
			=\ &R(\Delta_{m\times n})(R(\pi^1_{m,n}\times\pi^2_{m,n})(\varepsilon_m(x)_n(y)))\\
			=\ &R((\pi^1_{m,n}\times\pi^2_{m,n})\circ\Delta_{m\times n})(\varepsilon_m(x)_n(y))\\
			=\ &R(\pi^1_{m,n};\pi^2_{m,n})(\varepsilon_m(x)_n(y))\\
			=\ &R(\id_{m\times n})(\varepsilon_m(x)_n(y))\\
			=\ &\varepsilon_m(x)_n(y)
		\end{align*}
		So $\iota\circ\kappa=\id$ as well. This concludes the proof.
	\end{proof}
	\begin{remark}
		I came up with the construction for Theorem \ref{PSFCCC} myself, however I later found out that it is in fact just a special case of the something called the Day convolution. More about it \href{https://ncatlab.org/nlab/show/Day+convolution}{here}.
	\end{remark}
	With this, we can formulate the following theorem, which is basically the whole point of T-complexes:
	\begin{theorem}
		For every presheaf $P$ and $C\in\ASC_n$, one can define a natural transformation
		\[\xi:=P^{\iota_C}:P^{\Y(n)}\to P^{\X(C)}\]
		If $P$ is indeed a T-complex with fillers $\phi$, then $\xi:P^{\Y(n)}\twoheadrightarrow P^{\X(C)}$ is indeed a retraction whose right inverse can be given by:
		\begin{align*}
			\psi&:P^{\X(C)}\hookrightarrow P^{\Y(n)}\\
			\psi_m(c)&:=\sigma(\phi^{\tau(c)})
		\end{align*}
	\end{theorem}
	\begin{proof}
		Notice that each $\psi_m(c)$ is simply the filler of the simplicial complex $c:\X(C)\to P$ as defined by Theorem \ref{polyT}, so as a filler it lifts $c$, thus $\psi$ is indeed a componentwise right inverse to $\xi$, which is simply the precomposition with $\iota_C$. To show that $\psi$ is also natural, we simply show that the following diagram commutes
		\[\begin{tikzcd}
			\PSF(\X(C),P\circ\M(m))\arrow{rrr}{(\id_P*\M(f))\circ-}\arrow{d}[swap]{\psi_m}&&&\PSF(\X(C),P\circ\M(l))\arrow{d}{\psi_l}\\
			\PSF(\Y(n),P\circ\M(m))\arrow{rrr}[swap]{(\id_P*\M(f))\circ-}&&&\PSF(\Y(n),P\circ\M(l))
		\end{tikzcd}\]
		for every $f:l\to m$. This is the same as saying for every $c:\X(C)\to\M(m)$, the following diagram commutes:
		\[\begin{tikzcd}
			\X(C)\arrow[hook]{dr}[swap]{\iota_C}\arrow{r}{c}&P\circ\M(m)\arrow{rr}{\id_P*\M(f)}&&P\circ\M(l)\\
			&\Y(n)\arrow{u}[swap]{\sigma(\phi^{\tau(c)})}\arrow{rur}[swap]{\sigma(\phi^{\tau((\id_P*\M(f))\circ c)})}
		\end{tikzcd}\]
		This is however precisely the statement of Theorem \ref{GodementThom}.
	\end{proof}
	\end{comment}
\end{document}